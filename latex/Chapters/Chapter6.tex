\chapter{Conclusion}
\lhead{\emph{Conclusion}}  % Set the left side page header

%%%%%%%%%%%%%%%%%%%%%%%%%%%%%%%%%%%%
% Summary of Findings
%%%%%%%%%%%%%%%%%%%%%%%%%%%%%%%%%%%%

This research has examined the role of cyber currencies in cybercrime, focusing on how these digital assets are used in illicit activities and the steps taken by state actors to address this issue. According to the findings of this study, cyber currencies have facilitated a variety of illegal activities, including money laundering, ransomware attacks, drug trafficking, and many others. State actors have responded to this issue by implementing a variety of measures, including revising law enforcement strategies and regulatory frameworks to track and trace transactions. However, as cybercrime evolves and adapts to new technologies, it is likely that additional measures will be required to combat these types of crimes effectively.
%%%%%%%%%%%%%%%%%%%%%%%%%%%%%%%%%%%%
% Implications of Findings
%%%%%%%%%%%%%%%%%%%%%%%%%%%%%%%%%%%%

This study's findings have important implications for the development of cyber currencies and their potential adoption into mainstream usage. It is clear that effective measures are required to address the use of cyber currencies in illicit activities, while also ensuring the financial system's integrity and potential for innovation. 
%%%%%%%%%%%%%%%%%%%%%%%%%%%%%%%%%%%%
% Directions for Future Research
%%%%%%%%%%%%%%%%%%%%%%%%%%%%%%%%%%%%
There are several directions for future research in this area. One possible direction is to investigate the effectiveness of state actors' efforts to combat the use of cyber currencies in illegal activities. Another avenue of investigation is the potential impact of cyber currencies on the economy and their integration into mainstream financial systems. In addition, further research could examine the motivations and tactics of actors involved in these types of crimes, in order to better understand and anticipate their behavior. Overall, continued research and efforts will be critical in order to ensure the responsible and legitimate use of cyber currencies.

%%%%%%%%%%%%%%%%%%%%%%%%%%%%%%%%%%%%
% Version 2
%%%%%%%%%%%%%%%%%%%%%%%%%%%%%%%%%%%%

% State-driven cyber crime refers to criminal activities that are sponsored or supported by a government or state actor. Cyber currencies have increasingly been used as a tool in these types of crimes, providing a means of anonymous and untraceable transactions. The use of cyber currencies in state-driven cyber crime raises significant concerns and challenges, including the potential for damage to international relations and the reputational consequences of these crimes.

% In response to these challenges, a number of measures have been taken by law enforcement and regulatory agencies to combat state-driven cyber crime. These measures include the use of law enforcement efforts to track and seize cyber currency funds obtained through illicit means, as well as the implementation of regulations and laws to combat cyber crime. 

% The impact of state-driven cyber crime on the development of cyber currencies is complex and evolving. While cyber currencies offer the potential for faster and cheaper financial transactions, they also raise significant risks and challenges, including the potential for illicit use and abuse by state actors. It is important for the cyber currency industry and regulatory agencies to work together to address these challenges and ensure that cyber currencies can be used for legitimate purposes. 

% Overall, this study has examined the role of cyber currencies in state-driven cyber crime, including the types of cyber currencies used, the involvement of state actors, and the measures being taken to combat this type of crime. The findings of this study provide insight into the complex and evolving nature of this issue, and highlight the need for continued research and efforts to address the challenges and risks posed by state-driven cyber crime.
