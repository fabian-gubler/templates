\chapter{Introduction}
\lhead{\emph{Introduction}}  % Set the left side page header

% Cyber currencies, also known as cryptocurrencies, have gained significant attention in recent years due to their unique features and potential impact on the economy. These digital assets use cryptography for secure financial transactions and are facilitated through the use of a decentralized ledger, known as a blockchain. The anonymity and decentralized nature of cyber currencies have made them attractive to cybercriminals as a means of payment and exchange, leading to their use in a range of illicit activities such as money laundering, human trafficking, and drug trafficking (Böhme et al., 2015).

%%%%%%%%%%%%%%%%%%%%%%%%%%%%%%%%%
% Hook
%%%%%%%%%%%%%%%%%%%%%%%%%%%%%%%%%
Cyber currencies have attracted considerable attention in recent years due to their distinctive characteristics and potential economic impact. These virtual assets use cryptography to enable secure financial transactions and are facilitated through the use of a decentralized ledger, known as the blockchain. The anonymous and decentralized nature of cyber currencies has made them attractive for cybercriminals as a means of payment and exchange. In this regard, cyber currencies are facilitating a variety of illegal operations, such as money laundering, ransomware attacks, and drug trafficking \cite{bohme_bitcoin_2015} (p. 230).

%%%%%%%%%%%%%%%%%%%%%%%%%%%%%%%%%
% Relation to National Security
%%%%%%%%%%%%%%%%%%%%%%%%%%%%%%%%%
As stated in Satoshi Nakamoto's infamous white paper on bitcoin, one of the primary motivations for developing cyber currencies was to exist independently of a central authority or government \cite{nakamoto_bitcoin_nodate} (p. 4). Regardless of this intentional design decision, Bitcoin and similar currencies should not be viewed as separate from nation-states. Without a doubt, quite the contrary has been observed with regard to recent pressures to counteract its harmful influences in underground black markets and cybercrime \cite{ablon_markets_2014} (p. 4-6).

%%%%%%%%%%%%%%%%%%%%%%%%%%%%%%%%%
% State-driven Cyber Crime
%%%%%%%%%%%%%%%%%%%%%%%%%%%%%%%%%
Moreover, cyber currencies are increasingly being used to fund and support cybercrime, ranging from hacking and malware attacks to more sophisticated forms like economic espionage and sabotage \cite{reuter_information_2019} (p. 22). In particular, cyber currencies have been used as a tool in these kinds of crimes, because they provide the ability to create anonymous and untraceable transactions. 

%%%%%%%%%%%%%%%%%%%%%%%%%%%%%%%%%
\section{Research Questions}
%%%%%%%%%%%%%%%%%%%%%%%%%%%%%%%%%
The use of cyber currencies in cybercrime raises a number of critical issues and challenges that affect societies on a national and global scale. This thesis aims to address these questions through a review of relevant literature and analysis of case studies. The specific research questions for this study are: 
% TODO: Compare with Lecture Hints
\begin{enumerate}
    \item What types of cyber currencies are being used in cybercrime, and for what purposes?
    \item How are state actors involved in cybercrime, and what measures are being taken to combat them?
    \item How do state-driven actions influence the development of cyber currencies, and what are the future implications?
\end{enumerate}

%%%%%%%%%%%%%%%%%%%%%%%%%%%%%%%%%
\section{Structure of the Thesis}
%%%%%%%%%%%%%%%%%%%%%%%%%%%%%%%%%
To address these questions, the remaining chapters of this thesis are organized as follows:
% VERSION 2:2
The following chapter gives a definition of cyber currencies and a brief history of them, including the key features and traits that are unique to them. This will provide a foundational understanding for the rest of the thesis.
% VERSION 2:3 (Optional - Could merge with Ch. 2)
% In Chapter 2.2, the different types of cyber currencies used in state-sponsored cybercrime are explained, along with examples and their respective distinctions.
% VERSION 1:3
In Chapter 3, we examine how cyber currencies are used in cybercrime, including the use of specific cyber currencies.
% VERSION 1:4
Subsequently, Chapter 4 examines the measures taken by state actors to combat the use of cyber currencies in cybercrime, by looking at the challenges and examining regulatory frameworks as a potential solution.
Following, Chapter 4 investigates the measures taken by state actors to combat the use of cyber currencies in cybercrime by examining cyber currency specific challenges and discussing regulatory frameworks as a potential solution.
% VERSION 2:6
Further, Chapter 5 assesses the impact of cybercrime on the development of cyber currencies, by looking at according consequences for trust and innovation and the potential for increased regulation. 
% VERSION 2:7
Finally, Chapter 6 concludes with a summary of the study's key findings, as well as implications and suggestions for future research.


