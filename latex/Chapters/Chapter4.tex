\chapter{Approaches to Combat Illicit Use of Cyber Currencies}
\lhead{\emph{Approaches to Combatting the Illicit Use of Cyber Currencies}}  % Set the left side page header

%%%%%%%%%%%%%%%%%%%%%%%%%%%%%%%
\section{Challenges for Law Enforcement Agencies}
%%%%%%%%%%%%%%%%%%%%%%%%%%%%%%%

The anonymity of cyber currency transactions on the dark web makes it a popular choice for criminal activities, as it allows individuals and organizations to operate with a degree of protection from law enforcement. This anonymity, however, makes it difficult for authorities to track down and prosecute those engaged in illegal activities on the Dark Web. Furthermore, searching for criminal activity becomes an ongoing challenge for law enforcement because most search engines can only query results from the Clearnet \cite{lacson_21st_2016} (p. 45) \cite{zheng_learning_2013} (p. 201). 

To address these issues, law enforcement agencies have put in place a variety of measures, including the use of specialized software to track down cyber currency transactions. In addition, law enforcement is forming alliances with cybersecurity experts and other agencies to broaden their knowledge and expertise \cite{europol_internet_2017} (p. 40). However, due to the rapid evolution of cyber currencies and the sophisticated tactics used by cybercriminals, law enforcement agencies have found it difficult to keep up with the evolving threat. Concerning this issue, Christin suggested a number of ways to combat criminal activity on the dark web and has proposed "four intervention strategies that could be considered: disrupting the network, disrupting the financial infrastructure, disrupting the delivery model, and laissez-faire \cite{christin_traveling_2013} (p. 220).

The first option would involve shutting down the Tor network. This is nearly impossible because any computer with Tor installed can act as a node, with the software operating on a global scale. To put it differently, while a court order in one country may shut down a few nodes, removing a significant portion of nodes would require multinational cooperation \cite{dingledine_design_2006} (p. 4). 
The second option refers to the financial infrastructure of cyber currencies such as Bitcoin. This would involve an "attempt to manipulate the currency to create rapid fluctuations and impede transactions" \cite{christin_traveling_2013} (p. 221). While this is more feasible than stopping Tor, it would require significant financial investment and would most likely only be a temporary solution \cite{lacson_21st_2016} (p. 45). 
Third, another possible attack strategy is to disrupt the delivery model of illegal goods. That is, to strengthen controls at the post office and at customs to prevent illegal items from reaching their destination \cite{christin_traveling_2013} (p. 222). Finally, one last possible intervention strategy is to not intervene at all. Despite the fact that this is a politically questionable proposition, studies for instance show that preventing drug abuse is a far more cost-effective than enforcing drug prohibition \cite{christin_traveling_2013} (p. 222).

%%%%%%%%%%%%%%%%%%%%%%%%%%%%%%%
\section{Regulatory Frameworks}
%%%%%%%%%%%%%%%%%%%%%%%%%%%%%%%
As a whole, it was demonstrated that governmental law enforcement agencies have a number of options for combating the use of cyber currencies in cybercrime. Another approach has been to implement regulatory frameworks aimed at preventing illicit cyber currency  use in order to increase transparency in cyber currency transactions. These frameworks vary by country and take very different approaches. Some regulators may place an emphasis on consumer protection, while others focus on financial safety and integrity. \cite{narain_aditya_regulating_2022}. 

The Financial Action Task Force (FATF) recommendations is a prominent example of a widely accepted regulatory framework. Over 180 countries have endorsed the FATF recommendations \cite{financial_action_task_force_fatf_2012} (p. 7), which are universally recognized as the international standard for anti-money laundering efforts and countering the financing of terrorism. FATF moved quickly to provide a global framework for the regulation of virtual assets, including cyber currencies. These recommendations include provisions for the regulation of virtual asset service providers (VASPs), such as exchanges and wallet providers \cite{financial_action_task_force_fatf_2012} (p. 76). These in turn must put in place safeguards to ensure customer due diligence, record-keeping, and reporting of suspicious activity.

Besides the FATF recommendations, there are a number of other regulatory frameworks that address cyber currency use. These may include national laws and regulations, as well as international agreements. Some countries (including Japan and Switzerland) have amended or introduced new legislation governing cyber assets and service providers, while others (including the European Union, the United Arab Emirates, the United Kingdom, and the United States) are at the drafting stage \cite{narain_aditya_regulating_2022}. 

\section{Feasibility of Regulation}
In general, regulating cyber currency and virtual assets is a complicated and changing field. State actors are trying to find a balance between the possible benefits of these technologies and the need to stop them from being used illegally. The implementation of regulatory frameworks has received both praise and criticism. Supporters argue that these measures are necessary to protect consumers from fraud and to ensure the integrity of the financial system \cite{financial_action_task_force_fatf_2012} (p. 9). Critics, on the other hand, have expressed concern about regulatory frameworks' potential to stifle innovation and the growth of cyber currencies \cite{demertzis_economic_2018} (p. 10).

Furthermore, the feasibility of regulatory measures informs a secondary debate about legal ethics. Internet freedom has been fought for in courts since its inception. This due to it being a bastion of free speech and a tool for great good and innovation in the world  \cite{lacson_21st_2016} (p. 45). Andrew argues that when it comes to internet freedom, "it is impossible not to feel that any policing is regrettable"  \cite{andrew_internet_2010} (p. 1098). One such example is that while cyber currencies facilitate illegal activities, they also provide many liberties to people living under oppressive regimes by allowing them true anonymity.

%%%%%%%%%%%%%%%%%%%%%%%%%%%%%%%
% National Cyber Currencies
%%%%%%%%%%%%%%%%%%%%%%%%%%%%%%%

% Another approach has been the creation of national cryptocurrencies, which are digital currencies issued and backed by a national government. These currencies can be used as an alternative to traditional fiat currencies and may offer benefits such as increased speed and efficiency in financial transactions, as well as enhanced security. However, the use of national cryptocurrencies also raises concerns about centralization and the potential for governments to exert greater control over their citizens' financial activities.

% In addition to regulatory frameworks, some state actors have also developed national cyber currencies, which are digital assets issued and backed by the government (Central Bank of China, 2020). National cyber currencies are designed to provide a secure and transparent means of conducting financial transactions, while also allowing for the tracking and tracing of transactions to ensure compliance with relevant laws and regulations (Central Bank of China, 2020).

% The development of national cyber currencies has been met with both support and criticism. Supporters argue that these digital assets have the potential to provide a more secure and efficient means of conducting transactions and reduce the risk of cyber crime (Central Bank of China, 2020). Critics, on the other hand, have raised concerns about the potential for national cyber currencies to be used for surveillance and control (Buterin, 2014).

%%%%%%%%%%%%%%%%%%%%%%%%%%%%%%%
% \section{Effectiveness of Measures}
%%%%%%%%%%%%%%%%%%%%%%%%%%%%%%%

%%%%%%%%%%%%%%%%%%%%%%%%%%%%%%%%%%%%%
% State-driven
%%%%%%%%%%%%%%%%%%%%%%%%%%%%%%%%%%%%%

% State-driven cyber crime poses significant challenges and risks, including the potential for damage to international relations and the reputational consequences of these crimes. In response, a number of measures have been taken by law enforcement and regulatory agencies to combat this type of crime.

% One approach has been the use of law enforcement efforts to track and seize cyber currency funds that have been obtained through illicit means. For example, the U.S. Department of Justice has established a Cybercrime Unit specifically dedicated to investigating and prosecuting cyber crimes, including those involving cyber currencies (U.S. Department of Justice, n.d.). In addition, the U.S. Secret Service has a dedicated Cyber Intelligence Section that focuses on tracking and disrupting cyber crime, including the use of cyber currencies (U.S. Secret Service, n.d.).

% Another approach has been the implementation of regulations and laws to combat cyber crime, including state-driven cyber crime. For example, the U.S. has implemented the Bank Secrecy Act, which requires financial institutions to report suspicious activity, including the use of cyber currencies, to the Financial Crimes Enforcement Network (FINCEN) (U.S. Department of the Treasury, n.d.). In addition, the U.S. has implemented the Travel Rule, which requires financial institutions to report information about the sender and recipient of cross-border transactions over $3,000 (U.S. Department of the Treasury, n.d.).

% Overall, these measures are designed to disrupt and deter state-driven cyber crime, and to provide law enforcement with the tools and resources needed to track and prosecute these crimes. However, as cyber crime continues to evolve and adapt to new technologies, it is likely that additional measures will be needed to effectively combat this type of crime.

%%%%%%%%%%%%%%%%%%%%%%%%%%%%%%%%%%%%%
% Effectiveness of Measures
%%%%%%%%%%%%%%%%%%%%%%%%%%%%%%%%%%%%%

% It is difficult to determine the overall effectiveness of the measures taken by state actors to address the use of cyber currencies in cyber crime. Regulatory frameworks and national cyber currencies have the potential to provide a more secure and transparent means of conducting transactions and reduce the risk of cyber crime (Financial Action Task Force, 2018; Central Bank of China, 2020). However, the rapid evolution of cyber currencies and the sophisticated tactics used by cybercriminals have made it difficult for these measures to keep up with the evolving threat (Böhme et al., 2015).

% In addition, the implementation of regulatory frameworks and the development of national cyber currencies have been met with both support and criticism (Buterin, 2014). Some argue that these measures are necessary to ensure the integrity of the financial system and protect consumers from fraud and other illegal activities (Financial Action Task Force, 2018). Others have raised concerns about the potential for these measures to hinder innovation and stifle the growth of cyber currencies (Buterin, 2014).




