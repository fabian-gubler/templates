% \section{Cyber currencies' Impact on National Security}
%%%%%%%%%%%%%%%%%%%%%%%%%%%%%%%%%
% Decentralization
%%%%%%%%%%%%%%%%%%%%%%%%%%%%%%%%%
% First and foremost, the decentralized nature of cryptocurrency means that it operates outside the purview of traditional central banks and financial institutions. This has challenged the traditional financial systems of many nation states, as they struggle to understand and regulate the use of cryptocurrency within their borders. As Bank of England Governor Mark Carney noted in a speech at the Scottish Economics Conference, "The longer the portrayal myth of decentralization persists, the harder it will be to design effective regulation. If authorities do not act preemptively, cryptocurrencies could become more interconnected with the main financial system and become a threat to financial stability" (Carney, 2018).

%%%%%%%%%%%%%%%%%%%%%%%%%%%%%%%%%
% Anonymity
%%%%%%%%%%%%%%%%%%%%%%%%%%%%%%%%%
% Additionally, the anonymity afforded by certain cryptocurrency transactions has made it a popular means of exchange for illegal activities, such as money laundering and drug trafficking. This has led to increased pressure on governments to crack down on the use of cryptocurrency for illicit purposes, and to establish stricter regulations around its use. As the Financial Action Task Force (FATF) has noted, "Virtual assets, including virtual currencies, are vulnerable to money laundering and terrorist financing (ML/TF) risks, due to the anonymity they offer, their global reach, and the potential for cross-border transfers without the need for a financial institution" (FATF, 2019).

%%%%%%%%%%%%%%%%%%%%%%%%%%%%%%%%%
% Geopolitics
%%%%%%%%%%%%%%%%%%%%%%%%%%%%%%%%%
% Furthermore, the rapid growth and increasing mainstream adoption of cryptocurrency has led to a number of geopolitical tensions, as different nation states adopt different approaches to regulating and taxing cryptocurrency transactions. This has resulted in a patchwork of inconsistent regulations across the globe, which has led to debates about the appropriate role of cryptocurrency in the global financial system.

%%%%%%%%%%%%%%%%
% Chapter Conclusion
%%%%%%%%%%%%%%%%

% Overall, cyber currencies offer a number of potential benefits, including faster and cheaper transactions, increased financial inclusion, and greater control and autonomy for individuals and businesses (Böhme et al., 2015). However, they also raise a number of challenges and risks, including regulatory uncertainty, security concerns, and the potential for illicit use.

% it will continue to shape the global financial landscape in the coming years. has had a significant impact on nation states and geopolitical discussions.

%%%%%%%%%%%%%%%%%%%%%%%%%%%%%%%%%
% Bonus: Mining (Appendix?)
%%%%%%%%%%%%%%%%%%%%%%%%%%%%%%%%%

% Another key feature of cyber currency that has affected society on a global scale, is its process of creation.

% Bitcoins are created through a process known as mining, in which computers solve complex mathematical problems to verify transactions and add them to the blockchain (Zohar, 2019). 
% The process of mining requires a significant amount of computing power and is designed to become progressively more difficult over time, ensuring a controlled rate of inflation.

% In summary, mining is the process by which new Bitcoins are created and transactions are verified and added to the blockchain. The process requires a significant amount of computing power and is designed to become progressively more difficult over time, which helps to ensure a controlled rate of inflation for the cryptocurrency. However, the energy consumption of the mining process has led to concerns about its environmental impact.


% The high energy consumption of the mining process, which is necessary to verify transactions and add them to the blockchain, has been a particularly contentious issue. The media has generally reported that Bitcoin and other proof-of-work cryptocurrencies can have negative impacts on the environment. Some media outlets have reported that the energy consumption of the Bitcoin network is equivalent to that of entire countries, and that the carbon emissions resulting from this energy use are contributing to climate change. 

%%%%%%%%%%%%%%%%
% Example: Big Players
%%%%%%%%%%%%%%%%


% However, the Chinese government has also taken a cautious approach to cryptocurrency mining, and has implemented a number of measures to regulate and control the industry. In 2017, the Chinese government implemented a ban on initial coin offerings (ICOs) and shut down many cryptocurrency exchanges, which had a significant impact on the cryptocurrency industry in the country.

% In recent years, the Chinese government has also taken steps to curb the energy consumption of the cryptocurrency mining industry. In 2019, the government released a draft proposal that would limit the amount of electricity that could be used for cryptocurrency mining, and in 2021, the government announced a plan to phase out bitcoin mining in the country.

% % Not until Proof of stake
% In contrast, the proof-of-stake consensus mechanism used by Ethereum 2.0. is designed to be less energy-intensive and more environmentally friendly. Miners are chosen to validate transactions and add them to the blockchain based on the amount of cryptocurrency they hold, rather than on the amount of computing power they can provide. This can help to reduce the environmental impact of the cryptocurrency industry.

% Show graph


% References:

