\chapter{State Actions and the Development of Cyber Currencies}
\lhead{\emph{State Actions and the Development of Cyber Currencies}}  % Set the left side page header

Because cyber currencies are still a relatively new and rapidly evolving asset class, predicting the future with certainty is a difficult endeavor. One of the most pressing concerns about the future of cyber currencies is the fact that governments are increasingly and proactively regulating and monitoring their use and are successfully designing national regulatory solutions. \cite{ferreira_eu_2021} (p. 5). In more concrete terms, increased state efforts could result in stricter Know Your Customer and Anti-Money Laundering requirements in accordance with previously mentioned FATF recommendations.

In consequence, regulations may make it more difficult for exchanges and other institutions to facilitate the buying and selling of these digital assets, potentially limiting their adoption and use. Furthermore, associating cyber currencies with illegal activities may harm their reputation and public perception, leading to a lack of trust and adoption by individuals and businesses. This in term impedes the development of these digital assets, limiting their ability to disrupt traditional business models and create new opportunities. 

On the contrary, increased state efforts to combat cybercrime could also lead to greater trust in the security and stability of cyber currencies, leading to greater adoption and usage. According to Schaupp et al. \cite{schaupp_regulation_2022} (p. 16), an individual's attitude toward cyber currencies, such as perceived risk, innovativeness, and trust, significantly influences on an individual's intention to adopt cyber currencies for transactional usage. An increased sense of security and stability could especially be provided if state actors are successful in disrupting and dismantling major cyber criminal networks that have used cyber currencies to facilitate their illicit activities. This would imply that, while these efforts may pose challenges for the development of cyber currencies, they might also be required to ensure that these digital assets are used responsibly and benefit society as a whole.

Altogether, it is clear that state actors' efforts to combat cybercrime will have a significant impact on the development and adoption of these technologies. As a result, policymakers and industry stakeholders must carefully consider the potential unintended consequences of their actions to ensure that efforts to combat cybercrime do not stifle innovation or discourage legitimate use of these technologies.

%%%%%%%%%%%%%%%%%%%
% Bonus
%%%%%%%%%%%%%%%%%%%
% At the same time, Cyber currencies and blockchain technology, which underlies many of these digital assets, have the potential to bring about significant innovation and disruption in a number of areas. For example, the use of smart contracts, which are self-executing contracts with the terms of the agreement between buyer and seller being directly written into lines of code, could potentially streamline and automate a wide range of business processes.

% More generally, as cyber currencies are gaining wider acceptance, they are increasingly utilized in many legitimate industries, such as retail, and the financial sector \cite[p. 3]{andrew_internet_2010}. Many retailers and merchants are already accepting cyber currencies as a form of payment, and it is possible that this trend will continue to grow as more consumers become familiar with and comfortable using these digital assets. Additionally, cyber currencies enable the creation of new types of financial instruments and investment opportunities, such as tokenized assets, which are digital representations of physical assets that can be bought and sold using cyber currencies.


% Another important aspect is the process of "mining" cyber currencies, which involves using powerful computers to solve complex mathematical problems. This energy-intensive task has increasingly getting attention of nation states as it contributes to greenhouse gas emissions. As cyber currencies continue to gain mainstream acceptance and adoption, this could potentially lead to an increase in energy consumption and related environmental impacts.
