\chapter{Use of Cyber Currencies in Cyber Crime}
\lhead{\emph{Use of Cyber Currencies in Cyber Crime}}  % Set the left side page header

Cyber currencies, have been used to facilitate the sale of illegal goods and services. Although these transactions can be done using non-digital currencies, illicit sites are increasingly moving toward accepting only digital cyber currencies, because they enforce anonymity and security features \cite{ablon_markets_2014} (p. 11). 

The majority of illicit transactions are taking place on the dark web. This is due to its web architecture enhancing the security and encryption capabilities of its users, further enhancing the anonymity characteristics of cyber currencies. \cite{ablon_markets_2014}. The dark web contains content that has been intentionally concealed, typically including illegal and anti-social information, which "can only be accessed through specialized browsers such as Tor (short for The Onion Router)" \cite{weimann_going_2016} (p. 196).

\section{Overview of Cyber Currencies}

\subsection*{Bitcoin}
Bitcoin is the most widely used cyber currency in the sale of illegal drugs \cite{kim_get_2022} (p. 1). As the foremost and most well-known cyber currency, Bitcoin has a large user base and is accepted by a wide range of merchants, including those on the dark web. This makes it a convenient choice for individuals and organizations involved in illegal activities. In particular, Bitcoin as a currency is easily acquired, used, and exchanged, such as through the use of a Bitcoin ATM \cite{irwin_use_2016} (p. 414).

\subsection*{Monero}
Monero, a privacy-focused currency, is another cyber currency that has been used in black markets. Monero protects its users' confidentiality by employing advanced cryptographic techniques such as stealth addresses and ring signatures \cite{averin_review_2020} (p. 83). Regarding criminal activity, it has become a popular choice for ransomware attacks, because cybercriminals can demand payments in Monero without the risk of being traced.

\subsection*{Zcash}
Zcash is another cyber currency that is used in black markets and is often referred to as a more privacy-focused alternative to bitcoin. It is based on the same underlying technology as Bitcoin, but adds features that allow users to keep their transaction details private, by using a cryptographic technique called Zero Knowledge Proof \cite{harikrishnan_secure_2019} (p. 307). This makes Zcash especially appealing for use on the dark web. As a result, in instances where anonymity and privacy are essential, some marketplaces may even only accept Zcash as a payment method.

\subsection*{Implications for cybercrime}
It follows that cyber currencies, the dark web, and illegal activities are inextricably linked. More specifically, the anonymity and decentralization of cyber currencies make them ideal for the dark web, allowing users to conduct transactions without revealing their identities or leaving a traceable financial trail. More concretely, the anonymity and decentralization of cyber currencies make them a perfect fit for the dark web, as they allow users to make transactions without revealing their identities or leaving a traceable financial trail. Overall, Bitcoin is a popular choice for black market transactions due to its relative stability and liquidity compared to other cyber currencies. However, there is no agreement on which type of cyber currency is to be seen as the clear leader, as many currencies are interchangeable \cite{ablon_markets_2014} (p. 12).

\section{Types of Illegal activities}
There exists a number of ways in which cyber currencies are used to facilitate illegal activities. The following section concentrates on three major types of illegal activities that are closely related to cyber currencies: illicit drug sales, money laundering, and ransomware attacks. 

Throughout the section, a series of cases are presented to demonstrate how these activities are carried out using cyber currencies. These case studies will provide concrete examples of how cyber currencies are used in each of these illegal activities, allowing us to gain a better understanding of the motivations and mechanisms behind them. 

\subsection*{Illicit drug sales}

Cyber currencies are the primary form of payment on the dark web and have been used to facilitate the sale of illegal drugs. The Silk Road marketplace is one prominent example of illicit drug sales taking place on the dark web. The Federal Bureau of Investigation (FBI) shut down Silk Road in 2013, and its founder, Ross Ulbricht, was sentenced to life in prison. \cite{dolliver_evaluating_2015} (p. 1113).  According to the FBI's criminal complaint filed in the trial of Ross Ulbricht, the Silk Road market had nearly 150,000 buyers and nearly 4,000 vendors \cite{noauthor_united_nodate}. Despite its clearly unlawful operations, Silk Road provided an appealing value proposition to its users in the form of a close-knit community focused on the reliability of vendors and safe drug use \cite{lacson_21st_2016} (p. 46). One user, for example, stated that “relationships between vendors and consumers were based on levels of trust and professionalism” \cite{van_hout_silk_2013} (p. 387).

Regardless of this high-profile takedown, the use of cyber currencies in the sale of illegal drugs has continued to thrive on other dark web marketplaces. To illustrate, only two years later, the dark net aggregator DNStats.net listed 22 new Dark Net markets. Over 46,000 drugs were listed for sale on these markets, compared to 18,000 in October 2013, when Silk Road was shut down \cite{digital_citizens_alliance_silk_2015}. Moreover, it is expected that activity on the dark web and the use of cyber currencies will further increase in the coming years \cite{ablon_markets_2014} (p. 31).

\subsection*{Money Laundering}

Cyber currencies are frequently used to disguise the proceeds of illegal activities as legitimate funds, a process known as money laundering. Because cyber currencies are decentralized and barely regulated by any government or financial institution, they are an appealing vehicle used in money laundering.

Despite advancements in methods by which government institutions can reduce the ability to launder money, many mechanisms that provide anonymity continue to exist \cite{dupuis_money_2020} (p. 61). These “open door” technologies and products make it difficult for law enforcement to track and trace transactions involving cyber currencies. Additionally, the use of cyber currencies allows individuals to transfer funds across borders without going through traditional financial institutions \cite{filipkowski_cyber_2008} (p. 17), making money laundering even more difficult to detect and prevent.

There have been several cases where nation states have used cyber currencies to launder money and avoid economic sanctions. Venezuela, for instance, was sanctioned economically by the Trump Administration in 2018 for money laundering. These sanctions targeted the use of the country's own government-issued cyber currency, "Petro". \cite{noauthor_radical_nodate} (p. 1325). Petro emerged as an opportunity to raise new funds for the government and bypass US sanctions. It aspired to circumvent the existing barriers between Venezuelan financial operations and companies of the United States \cite{uzcategui_versatile_2020} (p. 193).

% Bonus: The case of Iran: In 2020, the U.S. Department of the Treasury's OFAC imposed economic sanctions on Iran and targeted the country's use of cryptocurrencies for money laundering and sanctions evasion. According to OFAC, Iran was using cryptocurrencies to "bypass U.S. economic sanctions and access the international financial system."

%%%%%%%%%%%%%%%%%%%%%%%%%%%%%%
% Bonus: ATMS
%%%%%%%%%%%%%%%%%%%%%%%%%%%%%%
% Another reason cyber currencies are used in money laundering is that they can be easily converted“open door”currency, such as US dollars or euros, through exchanges. This allows individuals to convert their illicit funds into a more widely accepted and usable form of currency.
  
% An example of this  is the use of cryptocurrency ATMs to launder money. According to a report published by Europol, "cryptocurrency ATMs are increasingly being used by money“launderers to convert illicit cash into virtual currency and vice versa." The report goes on to state that "criminals are attracted to the anonymity and lack of regulation of cryptocurrency ATMs, which makes them an attractive means of laundering illicit proceeds."

\subsection*{Ransomware Attacks}
Ransomware is a special type of malware which aims to encrypt a victim's files using strong
cryptography and demands a ransom from the victim to restore access to the files \cite{gonzalez_detection_2017} (p. 472).  Cybercriminals often demand payment in cyber currencies, in exchange for the decryption key to unlock the encrypted data.

The use of cyber currencies in ransomware attacks has enabled perpetrators to demand ransom payments in a difficult to trace manner, causing significant disruptions and financial losses for affected victims. Cyber currencies, such as Bitcoin, Monero, and Zcash, have had a significant impact on the proliferation of ransomware attacks. In 2021 alone, North America saw a 180 percent increase in ransomware attacks, while Europe saw an increase of 234 percent \cite{sonicwall_mid-year_2021} (p. 9). At the same time, the effective payments of ransomware attacks has risen significantly. In this case, up from 34 percent paid in 2020, 58 percent of ransomware-infected organizations agreed to pay a ransom in 2021 \cite{proofpoin”_2022_2022} (p.47).

One high-profile example of ransomware attacks is the WannaCry attack that occurred in May 2017. This cyberattack is considered the largest ransomware outbreak in history, impacting over 200,000 computers in 150 countries, extorting a relatively small sum of USD 140,000 in Bitcoin \cite{alraddadi_comprehensive_nodate} (p. 1, 5).

\subsection*{Other activities}
It is important to note that the use of cyber currencies for illegal purposes is not limited to these three activities. Cyber currencies also facilitate other types of illegal activity, such as fraud, cyber-enabled theft, and human trafficking. In addition, aside from black markets, cyber currencies, such as Bitcoin and others, have been used in a number of investment scams. These scams frequently involve deceptive investment schemes in which individuals are promised high returns in exchange for investing in virtual currencies. It is also worth noting  that while cyber currencies can be used to facilitate illegal activities, they are not inherently illegal and can also be used for legitimate purposes. Overall, because these transactions are anonymous, it can be difficult for law enforcement to track down and prosecute individuals involved in illegal activities. 