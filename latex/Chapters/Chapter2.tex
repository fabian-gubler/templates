\chapter{Background} \label{Chapter2}
\lhead{\emph{Background}}  % Set the left side page header


%%%%%%%%%%%%%%%%%%%%%%%%%%%%%%%%%
% Definition
%%%%%%%%%%%%%%%%%%%%%%%%%%%%%%%%%
Cyber currencies, also known as cryptocurrencies, are a peer-to-peer version of electronic cash that use cryptography to secure financial transactions \cite{nakamoto_bitcoin_nodate} (p. 1). They are decentralized, as they can be sent directly from one party to another without going through a financial institution, such as a bank or government. Instead, they rely on a distributed ledger technology known as a blockchain that enables transactions to be recorded and verified by a network of computers. \cite{universiti_utara_malaysia_robust_2018} (p. 24). In turn, this enables a secure and transparent transfer of value without the need for a centralized financial institution \cite{buterin_ethereum_nodate} (p. 34).

\section{History of Cyber Currencies}
%%%%%%%%%%%%%%%%%%%%%%%%%%%%%%%%%
% Early stage
%%%%%%%%%%%%%%%%%%%%%%%%%%%%%%%%%
The first digital currency, DigiCash was introduced in the late 1980s, which represented a new form of electronic money based on cryptographic protocols \cite{peters_trends_2015} (p. 4). However, it was not until the creation of Bitcoin in 2009 that cyber currencies received mainstream attention and adoption. Bitcoin was first introduced in a white paper published by an individual or group using the pseudonym Satoshi Nakamoto \cite{nakamoto_bitcoin_nodate}. The white paper, which was published shortly after the 2008 financial crisis, outlined a new system for electronic transactions in which no central authority is required to verify the validity of transactions.

One of the defining features of Bitcoin is that there are only 21 million bitcoins that can be added to the blockchain through a process called mining. As pointed out by Ethereum co-founder Vitalik Buterin, the economic scarcity means that Bitcoin is more than just a technological innovation \cite{ethereum_foundation_cryptoeconomics_2019}. He argued that Satoshi Nakamoto solved the pertinent economic incentives problem of digital currencies by creatively combining cryptography with economic assumptions through the limited supply of bitcoin.

% Reflected in the trading volume / economic interest
% OPTIONAL: Graph of price movement since inception
% Source: Authors using data from Blockchain.info and Quandl.com.

%%%%%%%%%%%%%%%%%%%%%%%%%%%%%%%%%
% Other crypto
%%%%%%%%%%%%%%%%%%%%%%%%%%%%%%%%%
Since the inception of Bitcoin, a number of new cyber currencies have emerged, each with its own set of distinct attributes. Some examples include Ethereum, which pioneered the concept of smart contracts; Monero, which prioritizes privacy and anonymity; and Litecoin, which aims to be a faster and more efficient version of Bitcoin. 

The rise of cyber currencies has been met with both enthusiasm and skepticism. Proponents argue that digital assets have the potential to revolutionize the financial industry by providing a more secure and efficient means of conducting transactions \cite{lu_blockchain_2019} (p. 83). Critics, on the other hand, have expressed concern about the potential use of cyber currencies in illicit activities, negative implications for the environment as well as the lack of regulatory oversight \cite{bohme_bitcoin_2015} (p. 214).

%%%%%%%%%%%%%%%%%%%%%%%%%%%%%%%%%
\section{Key Characteristics of Cyber Currencies} \label{2.1}
%%%%%%%%%%%%%%%%%%%%%%%%%%%%%%%%%
Despite the uncertainties involving cyber currencies, it is clear that they are affecting governments and societies in many groundbreaking ways. Cyber currencies have unique characteristics that distinguish them from traditional fiat currencies and pose new challenges for nation states at large. The following characteristics are especially to be taken into account for the rest of this thesis:

\begin{itemize}
    \item \textbf{Decentralization}: Cyber currencies are not controlled by central authority and consequently operate outside the purview of traditional central banks and financial institutions. Because of the decentralized nature, it is more difficult for law enforcement agencies to track and trace the origin and destination of cyber currency transactions. Furthermore, because there is no central authority that can block or interfere with transactions, decentralization provides greater resistance to censorship.
    \item \textbf{Anonymity}: Cyber currencies offer a high degree of anonymity, as direct personally identifiable information is omitted from any transaction \cite{ober_structure_2013} (p. 237). Because of the anonymity afforded by cyber currency transactions, it has become a popular method of payment and exchange for illegal activities such as ransomware extortion and DDoS attacks \cite{noauthor_internet_nodate} (p. 11). This characteristic makes it difficult for law enforcement to track down and prosecute criminals, as well as to attribute crimes to specific individuals, groups, or nation states.
    \item \textbf{Lack of regulation:} In many cases, cyber currencies are not subject to the same regulations and oversight as traditional financial systems. Furthermore, the rapid growth and increasing mainstream adoption of cyber currencies has led to a number of geopolitical tensions. Individual nations adopt different approaches when it comes to regulating and taxing cyber currency transactions \cite{law_at_sogang_university_school_of_law_jd_pittsburgh_current_2021} (p. 146).
\end{itemize}

% TODO: Check if all is covered
Given the decentralized, anonymous, and largely unregulated nature of cyber currencies, it is not surprising that cyber currencies have become a popular means of exchange in the world of cybercrime. Before delving into the role of state actors and their impact involving cybercrime, we must in the following chapter examine the various ways in which cyber currencies are used in cybercrime.