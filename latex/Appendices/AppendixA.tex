\chapter{Case studies of state-driven cyber crime involving cyber currencies}
\lhead{\emph{Case studies of state-driven cyber crime involving cyber currencies}}  % Set the left side page header

In order to gain a more comprehensive perspective, this study investigated several cases in which state-driven cybercrime is linked to cyber currencies.

\textbf{Case study 1: North Korea and the theft of cryptocurrency}

One of the most well-known examples of the use of cyber currencies in state-driven cyber crime is the use of Bitcoin by the North Korean government. North Korean hacker groups have been using cyber attacks to target financial institutions and cryptocurrency exchanges in order to steal funds and evade sanctions \cite{snyder_north_2022}. In particular, North Korea had launched at least five successful cyberattacks against cryptocurrency trades in Asia and had earned 571 million dollars \cite{kim_north_2022}

This case highlights the potential for state-driven cyber crime to impact the reputation and development of cyber currencies. The theft of such a large amount of cryptocurrency by a government raises concerns about the security and reliability of these technologies, and could potentially discourage mainstream adoption. 

\textbf{Case study 2: Syria and the use of cryptocurrency to fund terrorism}

Syria has been accused of using cryptocurrency to fund terrorism and evade sanctions \cite{lacson_21st_2016}. In particular, the Syrian government has allegedly used cryptocurrency to fund the activities of the Syrian Electronic Army, a state-sponsored hacking group. In general, a number of syrian individuals and entities have been identified to be using cryptocurrency to fund terrorism and support the Syrian government.

This case highlights the potential for state-driven cyber crime to involve the use of cryptocurrency to fund terrorism and evade sanctions. The use of these technologies in these types of crimes can have significant implications for national security, as they can facilitate the funding of illicit activities and the evasion of sanctions. 

\textbf{Case study 3: Iran and cyber attacks on financial institutions}

Iran has been accused of using cyber attacks to target financial institutions in a number of countries, including the United States and Saudi Arabia. Iran has used these attacks to steal funds and disrupt financial systems. In one high-profile case, Iranian hackers were accused of stealing approximately USD 81 million from the Bangladesh Bank's account at the Federal Reserve Bank of New York \cite{herman_cybersecurity_2021}

This case highlights the potential for state-driven cyber crime involving cyber currencies to have a significant impact on national security. The theft of such a large amount of funds by a government can have serious consequences for the stability of financial markets and the trust of consumers in these systems. In addition, the use of cyber currencies in state-sponsored cyber attacks may lead to increased regulation and scrutiny of these technologies.

\textbf{Case study 4: Russia and interference in the U.S. presidential election}

Another example of state-driven cyber crime involving cyber currencies is the Russian interference in the 2016 U.S. presidential election. Her, the Russian government used a campaign of hacking and disinformation in an attempt to influence the outcome of the election. As part of this campaign, Russian hackers targeted the emails of political campaigns and organizations, and released stolen emails through WikiLeaks \cite{heemsbergen_wikileaksorg_2021}

This case highlights the potential for state-driven cyber crime to have far-reaching consequences, including damage to international relations and the potential for increased regulation. The Russian interference in the U.S. election sparked widespread outrage and concern, and has led to increased scrutiny of cyber security and the role of cyber currencies in political campaigns. In addition, the use of cyber currencies in state-sponsored cyber crime may make it more difficult to trace and attribute these types of crimes, potentially impacting international relations.

